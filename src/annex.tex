%\bigskip

%\noindent {\large \textbf{Displaying a learning process of a SOM through SuperCollider via OSC}}

\addtocontents{toc}{\protect\setcounter{tocdepth}{0}}
\section{Displaying a learning process of a SOM \\through SuperCollider via OSC}
\label{ann:color}
\addtocontents{toc}{\protect\setcounter{tocdepth}{1}}

\bigskip

This code has been used to generate the illustrations of the figure \ref{fig:ex1} on page \pageref{fig:ex1} and its report is just about to give an idea about how to use N3 to mapping SOM and OSC protocol from N3 to SuperCollider -- mind that it works for limited SOM (100 neurons is fine -- that is to say $10\times10$), and
\textit{in fine}, the use of a graphical software like Processing\footnote{\href{https://processing.org/}{\texttt{\scriptsize https://processing.org/}}} will be more appropriate to display more complex configuration.

\bigskip

\begin{lstlisting}[language=SuperCollider]
/*
(create-mlt 'COLOR 3 100 :carte #'quadrare) 
;; or :carte #'rnd-map

;; to keep track of the initial value
(dolist (e (neurons-list COLOR)) 
   (setf (synapses-list e) (output e)))

;; retrieve initial state
;(dolist (e (neurons-list COLOR)) 
;   (setf (output e) (synapses-list e)))

;; set distance with modulo
;(setf (distance-in COLOR) 
;   (lambda* (a b) (euclidean a b :position t :modulo t)))

(let* ((num 3000) ;;  5 minutes with *latency* = 0.1
       (mvl (multiple-value-list (normalize-data 
          (loop for i in (neurons-list COLOR) 
             collect (synapses-list i)))))
       (dataset (car mvl))
       ;(fact (car (last mvl))) ; to retrieve initial dataset
       (initial-learning-rate 0.1)
       (initial-radius (/ (funcall #'mean 
          (list! (field COLOR))) 2))
       (initial-epoch (epoch COLOR)))
  (setf *latency* 0.1)
  (dotimes (k num)
    (setf (input COLOR)
          (nth (random (length dataset)) dataset)
          (radius COLOR)
          (exp-decay (1+ (- (epoch COLOR) initial-epoch)) initial-radius num 0.1)
          (learning-rate COLOR)
          (exp-decay (+ 1 (- (epoch COLOR) initial-epoch)) initial-learning-rate num 0.01))
    (learn COLOR)
    (send-udp (read-from-string 
       (format nil 
          "(\"\/N3-COLOR\" \"[~{[~{~S~^, ~}]~^, ~}]\")" 
          (loop for e in (neurons-list COLOR) 
             collect 
             (append (xpos (id e)) (output (id e))))))
       "127.0.0.1" 7771)
    (sleep *latency*)))
*/

(
var w, o, data, a = 10, b = 50;
w = Window("SOM activity: COLOR",
    Rect(rrand(100, 200), rrand(100, 200), a*b, a*b));
w.view.background = Color(0, 0, 0, 1);
w.onClose = {o.free};
w.front;
w.alwaysOnTop_(true);
thisProcess.openUDPPort(7771);
o = OSCFunc({ arg msg, time, addr, recvPort;
    data=msg[1];
    data=data.asString.interpret;
    {
		x = UserView(w, Rect(0, 0, a*b, a*b)).canFocus_(false);
		x.drawFunc = {
			Pen.use {
				Pen.width = 0.2;
				Array.fill(data.size,{|i|
					Pen.color = Color.new(
						data[i][2], 
						data[i][3], 
						data[i][4]);
	//Pen.addOval(Rect(data[i][0]*b, data[i][1]*b, b/2, b/2));
					Pen.addRect(Rect(
						data[i][0]*b, 
						data[i][1]*b, 
						b, b));
					Pen.perform(\fill);
				});
			};
		};
       x.refresh;
    }.defer;
}, '/N3-COLOR');
)
\end{lstlisting}
