%\bigskip
%\noindent {\large \textbf{Displaying a learning process of a SOM through SuperCollider via OSC}}

\addtocontents{toc}{\protect\setcounter{tocdepth}{0}}
\section{Displaying a learning process of a SOM \\through SuperCollider via OSC}
\label{ann:color}
\addtocontents{toc}{\protect\setcounter{tocdepth}{1}}

\bigskip

This code has been used to generate the illustrations of the figure \ref{fig:ex1} on page \pageref{fig:ex1} and its report is just about to give an idea of how to map SOM in N3 and how to manage OSC protocol from N3 to SuperCollider -- mind that it works for limited SOM (100 neurons is fine -- that is to say $10\times10$), and
\textit{in fine}, the use of a graphical software like Processing\footnote{\href{https://processing.org/}{\texttt{\scriptsize https://processing.org/}}} will be more appropriate to display more complex configuration.

\bigskip

\begin{lstlisting}[language=SuperCollider]
/*
(create-mlt 'COLOR 3 100 :carte #'quadrare) 
;; or :carte #'rnd-map

;; hack to keep track of the initial value
(dolist (e (neurons-list COLOR)) 
   (setf (synapses-list e) (output e)))

;; set distance with modulo
;(setf (distance-in COLOR) 
;   (lambda* (a b) (euclidean a b :position t :modulo t)))

(let ((num 3000) ;;  5 minutes with *latency* = 0.1
      (dataset (loop for i in (neurons-list COLOR) 
                 collect (synapses-list i)))
      (initial-learning-rate 0.1)
      (initial-radius (/ (funcall #'mean 
         (list! (field COLOR))) 2))
      (initial-epoch (epoch COLOR)))
  (setf *latency* 0.1)
  (dotimes (k num)
    (setf (input COLOR)
          (nth (random (length dataset)) dataset)
          (radius COLOR)
          (exp-decay (1+ (- (epoch COLOR) initial-epoch)) initial-radius num 0.1)
          (learning-rate COLOR)
          (exp-decay (+ 1 (- (epoch COLOR) initial-epoch)) initial-learning-rate num 0.01))
    (learn COLOR)
    (send-udp (read-from-string 
       (format nil 
          "(\"\/N3-COLOR\" \"[~{[~{~S~^, ~}]~^, ~}]\")" 
          (loop for e in (neurons-list COLOR) 
             collect 
             (append (xpos (id e)) (output (id e))))))
       "127.0.0.1" 7771)
    (sleep *latency*)))
*/

(
var w, o, data, a = 10, b = 50;
w = Window("SOM activity: COLOR",
    Rect(rrand(100, 200), rrand(100, 200), a*b, a*b));
w.view.background = Color(0, 0, 0, 1);
w.onClose = {o.free};
w.front;
w.alwaysOnTop_(true);
thisProcess.openUDPPort(7771);
o = OSCFunc({ arg msg, time, addr, recvPort;
    data=msg[1];
    data=data.asString.interpret;
    {
		x = UserView(w, Rect(0, 0, a*b, a*b)).canFocus_(false);
		x.drawFunc = {
			Pen.use {
				Pen.width = 0.2;
				Array.fill(data.size,{|i|
					Pen.color = Color.new(
						data[i][2], 
						data[i][3], 
						data[i][4]);
	//Pen.addOval(Rect(data[i][0]*b, data[i][1]*b, b/2, b/2));
	Pen.addRect(Rect(data[i][0]*b, data[i][1]*b, b, b));
					Pen.perform(\fill);
				});
			};
		};
       x.refresh;
    }.defer;
}, '/N3-COLOR');
)
\end{lstlisting}

\bigskip
\bigskip

\addtocontents{toc}{\protect\setcounter{tocdepth}{0}}
\section{Markov chain AREA instance \\for real time Music processing}
\label{ann:mcone}
\addtocontents{toc}{\protect\setcounter{tocdepth}{1}}

\subsection{Learning dataset -- rhizom-like network}

\bigskip

\noindent According to the \texttt{+corpus+} such as each sequence is defined as follow:
\begin{lstlisting}[language=N3]
((dur_1 deg_1 int_1) (dur_2 deg_2 int_2) 
   ... (dur_n deg_n int_n))
\end{lstlisting}
 Let initiate the area network with this corpus thus formatted.
\begin{lstlisting}[language=N3]
(create-mlt 'CS-INT 23 100)
(create-mlt 'CS-DEG 12 100)
(create-mlt 'CS-DUR 8 100)
(create-area 'CORPUS '(CS-DUR CS-DEG CS-INT))
\end{lstlisting}
Optionally set inner euclidean distance with modulo.
\begin{lstlisting}[language=N3]
(loop for i in (soms-list CORPUS) 
   do 
   (setf (distance-in (id i)) 
     (lambda* (a b) (euclidean a b :position t :modulo t))))
\end{lstlisting}
As seen previously in this writting, the mapping is done for each SOM according to their respective scaling using \glspl{lambda*} function.
\begin{lstlisting}[language=N3]
(mapping CS-INT 5000 (loop for i from -11 to 11 collect i) 
   :ds '(:bypass (lambda* (x) (>thrifty-code x 23 :mid 1)) 
   :update t))
(mapping CS-DEG 5000 (loop for i from 0 to 11 collect i) 
   :ds '(:bypass (lambda* (x) (>thrifty-code (mod x 12) 12 
      :left 0)) :update t))
(mapping CS-DUR 5000 (loop for i from 1 to 8 collect i) 
   :ds '(:bypass (lambda* (x) (>thrifty-code x 8 :right 1)) 
   :update t))
\end{lstlisting}
Then initiate the fanaux list according to the discrimination number defined apriori. This can be understood as the perception limits of the AI. 
\begin{lstlisting}[language=N3]
(update-fanaux CS-INT (loop for i from -11 to 11 
   collect (progn (setf (input CS-INT) 
      (>thrifty-code i (nbre-input CS-INT) :mid 1)) 
      (winner CS-INT))))
(update-fanaux CS-DEG (loop for i from 0 to 11 
   collect (progn (setf (input CS-DEG) 
      (>thrifty-code i (nbre-input CS-DEG) :left 0)) 
      (winner CS-DEG))))
(update-fanaux CS-DUR (loop for i from 1 to 8 
   collect (progn (setf (input CS-DUR) 
      (>thrifty-code i (nbre-input CS-DUR) :right 1)) 
      (winner CS-DUR))))
\end{lstlisting}
Make sure the discrimination is effective. 
\begin{lstlisting}[language=N3]
(loop for i from -11 to 11 collect 
   (progn (setf (input CS-INT) 
      (>thrifty-code i (nbre-input CS-INT) :mid 1)) 
      (winner-take-all (output (id (winner CS-INT))))))
(loop for i from 0 to 11 collect 
   (progn (setf (input CS-DEG) 
      (>thrifty-code i (nbre-input CS-DEG) :left 0)) 
      (winner-take-all (output (id (winner CS-DEG))))))
(loop for i from 1 to 8 collect 
   (progn (setf (input CS-DUR) 
      (>thrifty-code i (nbre-input CS-DUR) :right 1)) 
      (winner-take-all (output (id (winner CS-DUR))))))
\end{lstlisting}
The learning process at the area level consists to record all connected nodes from the cliques to the tournois according to the \texttt{cover-value} as the short term memory. Knowing that components of the corpus are partial sequences, only the onset state is enable.
\begin{lstlisting}[language=N3]
(loop for seq in (mapcar #'mat-trans +corpus+)
   do     
     (set-all-zeros CORPUS :mode :onset)  
     (loop
	      for i from 0 to (1- (length (car seq))) 
	      do
	      (loop
	        for s in (soms-list CORPUS)
	        for d in seq do
	        (setf (input (id s)) 
	              (scaling (nth i d) :mlt (id s))))
	      (learn CORPUS))
     (set-all-zeros CORPUS :mode :fine))
\end{lstlisting}

\subsection{Sequencing instantiation}

\begin{lstlisting}[language=N3]
(create-sequencing
 :net CORPUS
 :dub '*voice1*
 :tag "VOICE1"
 :port 7771)
(set-rule *voice1* 
  (make-list (length (soms-list (id (net self)))) 
    :initial-element '?))
(set-routine *voice1* 
  (markov-chain self :odds #'rnd-weighted))
;; ... and play
(act-routine *voice1*)
;; ... and stop
(kill-routine *voice1*)
\end{lstlisting}
