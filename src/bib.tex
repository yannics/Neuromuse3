\vspace{-1cm}
\renewcommand{\refname}{}
\makeatletter
\renewcommand\@biblabel[1]{}
\makeatother
\renewcommand\bibpreamble{$\rightarrow$ Here is the list of books, articles and  web pages that have been used in some way in the development of this work.\bigskip}

\begin{thebibliography}{99}
        		 
        \bibitem[-, 2012]{}Sous la direction de Daniel Andler. \textit{Introduction aux sciences cognitives}. Gallimard, 2012.
        \bibitem[-, 1995]{}Christophe Assens. \textit{Connexionnisme et théorie des organisations}. Cahiers de Recherche DMSP n°240, Novembre 1995.\\ \href{https://www.christophe-assens.fr/articles/cahier-de-recherche/}{\scriptsize{\texttt{https://www.christophe-assens.fr/articles/cahier-de-recherche/}}} \normalsize{}
        \bibitem[-, 2006]{}Hugues Bersini. \textit{De l'intelligence humaine à l'intelligence artificielle}. Ellipses, Paris 2006.
        \bibitem[-, 2012]{}Ludwig Von Bertalanfy. \textit{Théorie générale des systèmes}. Dunod, Paris 2012.
        \bibitem[-, 2011]{}Pierre Buser, Claude Debru. \textit{Le temps, instant et durée -- de la philosophie aux neurosciences}. Odile Jacob, Paris 2011.
        \bibitem[-, 2005]{}David Michael Cottle. \textit{Computer Music with examples in SuperCollider 3}. Web publication, 2005.\\ \href{http://rhoadley.net/courses/tech\_resources/supercollider/tutorials/cottle/CMSC7105.pdf}{\scriptsize{\texttt{http://rhoadley.net/courses/tech\_resources/supercollider/tutorials/cottle/CMSC7105.pdf}}} \normalsize{}
        \bibitem[-, 2009]{}Colloque annuel 2008 sous la direction de Stanislas Dehaene et Christine Petit. \textit{Parole et musique}. Odile Jacob, Paris 2009.
        \bibitem[-, 2002]{}Jean-Paul Delahaye. \textit{L'intelligence et le calcul -- de Gödel aux ordinateurs quantiques}. Belin, Pour la science 2002.
        \bibitem[-, 1974]{}Jean Fourastié. \textit{Comment mon cerveau s'informe -- Journal d'une recherche}. Robert Laffont, 1974.
                \bibitem[-, 1996]{}Peter Hadreas. \textit{Searle versus Derrida ?}. Philosophiques, Volume 23, n°2, 1996, pp. 317-326. Traduction Josette Lanteigne.\\ \href{https://doi.org/10.7202/027399ar}{\scriptsize{\texttt{https://doi.org/10.7202/027399ar}}} \normalsize{}

        \bibitem[-, 2014]{}Arslan Hamza Cherif. \textit{Neurogrid : un circuit intégré pour simuler 1 million de neurones}. Web publication, 2014.\\ \href{https://www.developpez.com/actu/70629/Neurogrid-un-circuit-integre-pour-simuler-1-million-de-neurones-9-000-fois-mieux-qu-une-simulation-sur-ordinateur/}{\scriptsize{\texttt{https://www.developpez.com/actu/70629/}}} \normalsize{}
        \bibitem[-, 2004]{}Jocelyne Kiss. \textit{Composition musicales et sciences cognitives - Tendances et perspectives}. L'Harmattan, 2004.
        \bibitem[-, 2012]{}Rémy Lestienne. \textit{Dialogues sur l'émergence}. Le Pommier, Paris 2012.
        \bibitem[-, 2011]{}Pierre Mercklé. \textit{Sociologie des réseaux sociaux}. La découverte, Paris 2011.
        \bibitem[-, 2008]{}Alp Mestan. \textit{Introduction aux Réseaux de Neurones Artificiels Feed Forward}. Web publication, 2008.\\ \href{https://alp.developpez.com/tutoriels/intelligence-artificielle/reseaux-de-neurones/}{\scriptsize{\texttt{https://alp.developpez.com/tutoriels/intelligence-artificielle/reseaux-de-neurones/}}} \normalsize{}
        \bibitem[-, 1994]{}Israel Rosenfield. \textit{L'invention de la mémoire}. Flammarion, 1994.
        \bibitem[-, 2011]{}Denis Shasha, Cathy Lazere. \textit{Quand la vie remplace le silicium -- Aux frontières de la bio-informatique}. Dunod, Paris 2011.
        \bibitem[-, 1989]{}Rupert Sheldrake. \textit{La mémoire de l'univers}. Édition du Rocher, 1989.
        \bibitem[-, 2012]{}Rupert Sheldrake. \textit{Réenchanter la science}. Albin Michel, Paris 2012.
        \bibitem[-, 2001]{sn} Bob Snyder, Robert Snyder. \textit{Music and Memory: An Introduction}. A Bradford Book Mit Press, 2001.\\ \href{https://monoskop.org/images/f/f3/Snyder\_Bob\_Music\_and\_Memory\_An\_Introduction.pdf}{\scriptsize{\texttt{https://monoskop.org/images/f/f3/Snyder\_Bob\_Music\_and\_Memory\_An\_Introduction.pdf}}} \normalsize{}
        \bibitem[-, 2014]{}Richard Solé, Bernat Corominas Murtra, Jordi Fortuny. \textit{La structure en réseaux du langage}. Dossier hors-série Pour la science, n°82 janvier/Mars 2014, pp. 8-15.        
        \bibitem[-, 2005]{}Caroline Tourbe. \textit{Nous avons un deuxième cerveau!}. Science \& vie, n°1058 novembre 2005, pp. 64-79.
        \bibitem[-, 2016]{}Andrea Valle. \textit{Introduction to SuperCollider}. The MIT Press, 2016.\\ \href{https://www.scribd.com/document/381922613/intro-to-SuperCollider-pdf}{\scriptsize{\texttt{https://www.scribd.com/document/381922613/intro-to-SuperCollider-pdf}}} \normalsize{}
        \bibitem[-, 2011]{}Scott Wilson, David Cottle, Nick Collins. \textit{The SuperCollider Book}. The MIT Press, 2011.\\ \href{https://www.scribd.com/doc/296526523/The-Supercollider-Book-Scott-Wilson-David-Cottle-Nick-Collins}{\scriptsize{\texttt{https://www.scribd.com/doc/296526523/}}} \normalsize{}
                
        \bibitem[-, 2016]{}Les Dossiers de Sciences \& Univers. \textit{Votre cerveau va vous étonner!}. n°4 novembre 2015/janvier 2016
        
\end{thebibliography}